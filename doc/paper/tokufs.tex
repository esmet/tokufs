\documentclass[12pt] {article}

\begin {document}
\title {TokuFS: A streaming B-tree based filesystem}
\author {John Esmet}

\maketitle

\abstract {
TokuFS is a fragmentation-free, microdata optimized file system.
}

\section* {Introduction}
File systems suck at random writes. TokuFS does not. Random writes
are not uncommon. Consider many processes writing out different
files concurrently. Normal file systems will either write out
the data by offset as the requests come in, which means random IO,
or they write out the blocks where the disk head is currently,
resulting in fragmentation for at least one party. Neither
are optimal.

\section* {Design overview}

TokuDB is used as the underlying storage system, with one database for 
metadata and one for file data. The block database uses the file path as the
beginning of its key with the block number as the last part, which means
datablocks for a single file are sorted in order. Both metadata and blocks
are stored in \emph{level order}, which means files and directories sharing 
a parent are adjacent. Reading a directories contents are always fast, no
matter how many entries the directory has. Similarly, reading a file
sequentially is always fast, no matter how conjested the file system is
or how the datablocks were written out, or if other files were written
concurrently.

\section* {Preliminary results}

A benchmark in which 30GB of file data was written out in random 575 byte
segments shows a 160x speedup over a traditional file system such as XFS.

\section* {Known issues and future work}

Because of the invariant that datablocks are never fragmented, it is necessary
to move an entire subtree of data when a directory is renamed. Even those
deletes are fast in TokuDB, it is still an overwhelming amount of data to 
delete for something as simple as a directory rename. TokuDB may support
broadcast key updates or O(1) subtree swaps in the future, solving this
issue gracefully.

\end {document}
